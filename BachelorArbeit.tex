%------------------------------------------------------------------------------
%---------------- Docmentenklasse, Layout und Ränder: -------------------------
%------------------------------------------------------------------------------

\documentclass[paper=a4,                % Papierformat DIN A4
                BCOR=12mm,              % 12mm Binderandkorrektur
                parskip=half,           % Absätze als halbe Leerzeile
                cleardoublepage=plain,  % Keine Kopf/Fußzeile auf Leerseiten
                bibliography=totoc,     % Bibliographie als nicht-nummeriertes 
                                        % Kapitel im Inhaltsverzeichnis
                ngerman,                % Deutsche Spracheinstellung
                openany,                % Kapitel dürfen auf beiden Seiten beginnen
                captions=tableheading,  % Passt das Spacing an Captions über Tabellen an
                headsepline,            % Linie unter der Kopfzeile
                titlepage=firstiscover, % Titelseite ist Deckblatt, symmetrische Ränder
                headings=normal         % kleinere Überschriften
            ]
            {scrbook}


% Beschränkung auf chapter und section im Inhaltsverzeichnis:            
\setcounter{tocdepth}{1}

%\usepackage[protrusion=true, expansion]{microtype}

%------------------------------------------------------------------------------
%----------------------------- Encoding: --------------------------------------
%------------------------------------------------------------------------------

\usepackage[utf8]{luainputenc}

%------------------------------------------------------------------------------
%------------------------------ Sprache und Schrift: --------------------------
%------------------------------------------------------------------------------

\usepackage[ngerman]{babel}
\usepackage{csquotes}           % stellt den \enquote{} Befehl
\usepackage{lmodern}            % Schriftart Latin Modern


%------------------------------------------------------------------------------
%-------------------------Pakete für Kopf/Fußzeile: ---------------------------
%------------------------------------------------------------------------------
% Einkommentieren und definieren, falls man mehr als den Koma.Script Header
% möchte, nicht empfohlen
% \usepackage{scrpage2}
% \pagestyle{scrheadings}


%------------------------------------------------------------------------------
%------------------------ Für die Matheumgebung--------------------------------
%------------------------------------------------------------------------------

\usepackage{amsmath}
\usepackage{amssymb}
\usepackage{mathtools}
\usepackage{xfrac}          % schöne Brüche im Text mit \sfrac{}{}
\usepackage[ngerman]{varioref}


%	Gleichungsnummern Kapitel.Unterkapitel.Gleichung
\renewcommand{\theequation}{\thesection{}.\arabic{equation}}
\numberwithin{equation}{chapter}
\numberwithin{equation}{section}

%------------------------------------------------------------------------------
%---------------------------- Einheiten ---------------------------------------
%------------------------------------------------------------------------------

\usepackage[locale=DE, separate-uncertainty=true, per-mode=fraction]{siunitx}

%------------------------------------------------------------------------------
%------------------------------ Tabellen: -------------------------------------
%------------------------------------------------------------------------------


\usepackage{booktabs}
\usepackage{tabulary}
\usepackage{threeparttable}

%------------------------------------------------------------------------------
%------------------------------ Grafiken: -------------------------------------
%------------------------------------------------------------------------------

\usepackage[]{graphicx}
\usepackage{captcont}
\usepackage[labelfont=bf]{caption} % das Abbildung x: und Tabelle x:  fett
\usepackage{subcaption}


%------------------------------------------------------------------------------
%------------------------------ Bibliographie ---------------------------------
%------------------------------------------------------------------------------

\usepackage[backend=biber]{biblatex}
    \addbibresource{references.bib}     % die Bibliographie einbinden

% fix biblatex stuff (add collaboration, et al. etc)
\DeclareSourcemap{
 \maps[datatype=bibtex,overwrite=true]{
  \map{
    \step[fieldsource=Collaboration, final=true]
    \step[fieldset=usera, origfieldval, final=true]
  }
 }
}
\renewbibmacro*{author}{\iffieldundef{usera}{\printnames{author}}{\printfield{usera}\addcomma\addspace\printnames{author}}}
\DefineBibliographyStrings{ngerman}{andothers = {{et\,al\adddot}}}


%------------------------------------------------------------------------------
%------------------------------ Sonstiges: ------------------------------------
%------------------------------------------------------------------------------


\usepackage[pdfusetitle,unicode]{hyperref}
\usepackage{xcolor}
\usepackage{eurosym}


%------------------------------------------------------------------------------
%-------------------------    Angaben zur Arbeit   ----------------------------
%------------------------------------------------------------------------------

%Titel der Arbeit
\newcommand{\thetitle}{\LaTeX-Vorlage für die Bachelorarbeit}
\newcommand{\Jahr}{2014}
\newcommand{\Geburtsort}{Castrop-Rauxel}
\newcommand{\Lehrstuhl}{Experimentelle Physik V}
\newcommand{\Betreuer}{Prof. Dr. Erstgutachter}
\newcommand{\Zweitgutachter}{Prof. Dr. Zweitgutachter}
\newcommand{\Abgabedatum}{11. Juli 2014}

%Author und Email-Adresse
\author{Maximilian Nöthe \thanks{maximilian.noethe@tu-dortmund.de}}

\title{\thetitle}
\date{\Jahr}

\subject{Arbeit zur Erlangung des akademischen Grades\\Bachelor of Science}
\publishers{Lehrstuhl für \Lehrstuhl \\ Fakultät Physik \\ Technische Universität Dortmund}

%Gutachterseite
\lowertitleback{
    \begin{tabbing}
        Erstgutachter: \hspace{3em}\=   \Betreuer \\ 
        Zweitgutachter: \> \Zweitgutachter\\
        Abgabedatum: \>\Abgabedatum
    \end{tabbing}
}




\begin{document}
\frontmatter
\maketitle
% hier beginnt der Vorspann, nummeriert in römischen Zahlen
\thispagestyle{plain}
\section*{Kurzfassung}

\textbf{\large Empfohlen wird die Verwendung dieser Vorlage mit der jeweils aktuellsten TeXLive Version (Linux, Windows) bzw. MacTeX Version (MacOS).
Aktuell ist dies TeXLive 2014. Download hier: 
}

\href{https://www.tug.org/texlive/}{\textbf{\large https://www.tug.org/texlive/}}

\textbf{\large
Wichtig ist auch, dass die Source-Dateien UTF-8 kodiert sind. Dies
ist nur unter Windows ein Problem, benutzen Sie einen Editor, der
UTF-8 unterstützt (z.B. TexMaker ab V4, notepad++, sublime).
}

Eine aktuelle Version dieser Vorlage gibt es unter 

\href{https://github.com/MaxNoe/VorlageBachelorArbeit/tree/tu-farben}{www.github.com/MaxNoe/VorlageBachelorArbeit/tree/tu-farben}.

Eine farblich neutrale Variante steht unter  

\href{https://github.com/MaxNoe/VorlageBachelorArbeit}{www.github.com/MaxNoe/VorlageBachelorArbeit}

zur Verfügung.

Falls es Probleme mit der Vorlage gibt, einfach ein \emph{Issue} auf GitHub aufmachen oder eine Email an
\href{mailto:maximilian.noethe@tu-dortmund.de}{maximilian.noethe@tu-dortmund.de} schreiben.


Hier steht eine Kurzfassung der Arbeit in deutscher Sprache inklusive der Zusammenfassung der
Ergebnisse.
Zusammen mit der englischen Zusammenfassung muss sie auf diese Seite passen.

\section*{Abstract}

The abstract is a short summary of the thesis in English, together with the German summary it has to fit on this page.

\tableofcontents

\mainmatter
% Hier beginnt der Inhalt mit Seite 1 in arabischen Ziffern
\input{./Inhalt/01_struktur.tex}
\chapter{Wichtige Hinweise und \LaTeX-Grundlagen}

Diese Vorlage ist auf die Kompilierung mit \texttt{lualatex} ausgelegt. Es wird die \KOMAScript-Klasse \texttt{scrbook} verwendet.
Falls Sie Änderungen am Layout vornehmen möchten, lesen Sie die \KOMAScript-Dokumentation: \cite{koma}.

Lesenswert ist außerdem das \LaTeX-Tabu: \cite{l2tabu}, sowie \emph{Modern Packages for \LaTeX} von Philipp Leser: \cite{pleser}.

\section{Make}\label{make}

Für diese Vorlage wird ein Makefile zur Verfügung gestellt, welches automatisch alle Schritte ausführt, die für das fertige Dokument nötig sind. Make prüft, ob die Quelldateien verändert wurden, falls nicht, werden auch keine Befehle ausgeführt.

Folgende Befehle werden durch das Makefile druchgeführt, falls sich die Quelldateien verändert haben:

\begin{enumerate}
    \item \texttt{lualatex BachelorArbeit.tex}
    \item \texttt{biber BachelorArbeit.tex}
    \item \texttt{lualatex BachelorArbeit.tex}
    \item \texttt{lualatex BachelorArbeit.tex}
    \item verschieben der Hilfs- und Logdateien in den Ordner logfiles
\end{enumerate}

Download und weitere Informationen zu Make gibt es unter \cite{make}.
Rufen Sie einfach in der Konsole im Verzeichnis der Arbeit den Befehl \texttt{make}.

\section{\LaTeX-Grundlagen}

Bitte beachten Sie beim schreiben der Arbeit folgende Konventionen bzw. Grundlagen:

\begin{itemize}
    \item \textbf{Abschnitte und Zeilenumbrüche} \\
        Es sollten im Fließtext keine Zeilenumbrüche mit \textbackslash\textbackslash \ erzwungen werden.
        Schreiben Sie höchsten einen Satz in eine Code-Zeile.
        Absätze werden im Code mit einer Leerzeile markiert und dann entsprechend der Einstellung von \texttt{parskip} in der Dokumentenklasse gesetzt.
    \item \textbf{Trennung} \\
        \LaTeX \ trennt keine Wörter, in denen Umlaute oder Bindestriche vorkommen. 
        In diesen Wörtern müs\-sen die mög\-lichen Trennstellen mit einem \texttt{\textbackslash -} markiert werden: \texttt{Wör\textbackslash-ter}.
        Andernfalls können diese Wörter dann in den Rand hineinragen.


\end{itemize}

\section{Zahlen und Einheiten}

Jede Zahl, jede Einheit und jede Zahl mit Einheit sollte mit Hilfe der in dem Paket \texttt{siunitx} zur Verfügung gestellten Befehle gesetzt werden.

Grundsätzlich gilt: Einheiten werden aufrecht gesetzt und haben ein kleines Leerzeichen (\verb+\,+) Abstand zu ihrer Zahl. 
Werden Fließkommazahlen ohne \texttt{siunitx} gesetzt, entsteht ein hässlicher Leerraum zwischen Komma und erster Nachkommastelle, da \LaTeX \ das Komma nicht als Dezimaltrennzeichen, sondern als Satzzeichen interpretiert.

Das Paket wurde mit deutschen Spracheinstellungen (also mit Komma als Dezimaltrennzeichen und $\cdot$ zwischen Zahl und Zehnerpotenz) geladen, sowie mit den Einstellungen, dass die Standardabweichung stets durch $\pm$ abgetrennt wird und Einheiten falls nötig als Brüche ausgegeben werden.

\begin{table}[!h]
    \centering
    \caption{Beispiele für siunitx}
    \label{tab:si}
    \begin{tabular}{l r}
        \toprule
        Befehl     &   Ergebnis \\
        \midrule
        \verb+\num{1.2345}+ & \num{1.2345} \\
        \verb+\num{1.2e3}+ & \num{1.2e3} \\
        \verb_\num{1.2 +- 0.2}_ & \num{1.2+-0.2} \\
        \verb+\num{10000}+ & \num{10000} \\
        \verb+\si{\meter\per\second}+ & \si{\meter\per\second} \\
        \verb+\SI{1.2(1)}{\micro\ampere}+ & \SI{1.2(1)}{\micro\ampere} \\
        \verb+\SI{1.2\pm0.1e3}{\kilo\gram\per\cubic\meter}+ & \SI{1.2\pm0.1e3}{\kilo\gram\per\cubic\meter} \\
        \bottomrule 
    \end{tabular}
\end{table}

Das Paket stellt unter anderem die drei wichtigen Befehle
\begin{itemize}
    \item \texttt{\textbackslash num\{Zahl\}},
    \item \texttt{\textbackslash si\{Einheit\}} und
    \item \texttt{\textbackslash SI\{Zahl\}\{Einheit\}}
\end{itemize}
zur Verfügung.
Diese Befehle sollten stets genutzt werden, wenn Zahlen angegeben werden. 
Sie funktionieren sowohl im Text- als auch im Mathematikmodus.
In Tabelle \ref{tab:si} sind einige Beispiele aufgetragen. Bitte lesen Sie die Dokumentation \cite{siunitx}.

\section{Das Literaturverzeichnis}

Das Literaturverzeichnis wird mit Hilfe von BibLaTeX und biber erstellt.
Tragen Sie alle ihre Quellen in die Datei \texttt{references.bib} ein, Sie enthält bereits
einige Beispiele. Für weitere Informationen lesen Sie bitte die Dokumentation \cite{biblatex}.

Im Text können Sie mit \verb_\cite{kürzel}_ zitieren. Seitenzahlen geben Sie in eckigen Klammern an:
\verb_\cite[S.~10]{kürzel}_. Das Literaturverzeichnis ist so eingestellt, dass es Ihre Quellen in alphabetischer Reihenfolge nummeriert.

Damit das Literaturverzeichnis erstellt wird, ist ein Aufruf von \texttt{biber} nach einem ersten kompilieren mit \texttt{lualatex} nötig. Danach muss das Dokument erneut mit \texttt{lualatex} kompiliert werden. Haben Sie Make installiert, übernimmt dies für Sie das Makefile. Siehe Sektion \ref{make}

\input{./Inhalt/03_figs_tabs.tex}

\appendix
% Hier beginnt der Anhang, nummeriert in lateinischen Buchstaben
\chapter{Ein Anhangskapitel}

Hier könnte ein Anhang stehen.


\backmatter
\printbibliography

\cleardoublepage
\newpage
\thispagestyle{empty}
\section*{Eidesstattliche Versicherung}
Ich versichere hiermit an Eides statt, dass ich die vorliegende Bachelorarbeit mit dem Titel \enquote{\thetitle} selbst\-ständig und ohne unzulässige fremde Hilfe erbracht habe.
Ich habe keine anderen als die angegebenen Quellen und Hilfsmittel benutzt, sowie wörtliche und sinngemäße Zitate kenntlich gemacht. 
Die Arbeit hat in gleicher oder ähnlicher Form noch keiner Prüfungsbehörde vorgelegen.

\vspace*{1cm}

\rule{0.4\linewidth}{0.25pt}  \hfill \rule{0.4\linewidth}{0.25pt}\\
Ort, Datum \hfill Unterschrift\hspace*{9.1em}\\

\subsection*{Belehrung}
Wer vorsätzlich gegen eine die Täuschung über Prüfungsleistungen betreffende Regelung einer Hochschulprüfungsordnung verstößt, handelt ordnungswidrig.
Die Ordnungswidrigkeit kann mit einer Geldbuße von bis zu \SI[round-mode=places, round-precision=2]{50000}{\officialeuro} geahndet werden. 
Zuständige Verwaltungsbehörde für die Verfolgung und Ahndung von Ordnungswidrigkeiten ist der Kanzler/die Kanzlerin der Technischen Universität Dortmund. 
Im Falle eines mehrfachen oder sonstigen schwerwiegenden Täuschungsversuches kann der Prüfling zudem exmatrikuliert werden \mbox{(\S63 Abs. 5 Hochschulgesetz -HG-).}

Die Abgabe einer falschen Versicherung an Eides statt wird mit Freiheitsstrafe bis zu 3 Jahren oder mit Geldstrafe bestraft.

Die Technische Universität Dortmund wird ggf. elektronische Vergleichswerkzeuge (wie z.B. die Software \enquote{turnitin}) zur Überprüfung von Ordnungswidrigkeiten in Prüfungsverfahren nutzen.

Die oben stehende Belehrung habe ich zur Kenntnis genommen.

\vspace*{1cm}

\rule{0.4\linewidth}{0.25pt}  \hfill \rule{0.4\linewidth}{0.25pt}\\
Ort, Datum \hfill Unterschrift\hspace*{9.1em}\\



\end{document}
