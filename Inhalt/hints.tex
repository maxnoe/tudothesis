\thispagestyle{empty}
\setcounter{page}{2}
\section*{Hinweise}
Empfohlen wird die Verwendung dieser Vorlage mit der jeweils aktuellsten TeXLive Version (Linux, Windows) bzw. MacTeX Version (MacOS).
Aktuell ist dies TeXLive 2015. Download hier:
\begin{center}
  \href{https://www.tug.org/texlive/}{https://www.tug.org/texlive/}
\end{center}
Bei Verwendung von TexLive Versionen 2014 und älter sollte
die Zeile \verb+\RequirePackage{fixltx2e}+ als erste Zeile der Präambel
noch vor der Dokumentenklasse eingefügt werden.
Dies lädt diverse Bugfixes von LaTeX, die ab TexLive 2015 Standard sind.

Achten Sie auch auf die Kodierung der Quelldateien.
Bei Verwendung von Xe\LaTeX\ oder Lua\LaTeX\ (empfohlen) müssen die
Quelldateien UTF-8 kodiert sein.
Bei Verwendung von pdf\LaTeX\ nutzen Sie die Pakete \texttt{inputenc} und \texttt{fontenc} mit der korrekten Wahl der Kodierungen.

Eine aktuelle Version dieser Vorlage steht unter 
\begin{center}
  \href{https://github.com/MaxNoe/TUDoThesis}{www.github.com/MaxNoe/TUDoThesis}
\end{center}
zur Verfügung.

Für Rückmeldungen und bei Problemen mit der Klasse oder Vorlage, bitte ein \emph{Issue} auf GitHub aufmachen oder eine Email an
\href{mailto:maximilian.noethe@tu-dortmund.de}{maximilian.noethe@tu-dortmund.de} schreiben.

Wenn Sie die Dokumentenklasse mit der Option \texttt{tucolor} laden, werden verschiedene Elemente in TU-Grün gesetzt.
