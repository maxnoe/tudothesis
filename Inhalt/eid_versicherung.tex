\thispagestyle{empty}
\section*{Eidesstattliche Versicherung}
Ich versichere hiermit an Eides statt, dass ich die vorliegende Abschlussarbeit mit dem Titel \enquote{\thetitle} selbstständig und ohne unzulässige fremde Hilfe erbracht habe.
Ich habe keine anderen als die angegebenen Quellen und Hilfsmittel benutzt, sowie wörtliche und sinngemäße Zitate kenntlich gemacht. 
Die Arbeit hat in gleicher oder ähnlicher Form noch keiner Prüfungsbehörde vorgelegen.

\vspace*{1cm}\noindent
\begin{center}
  \begin{tabular}{@{}p{0.4\textwidth}@{\hspace{0.15\textwidth}}p{0.4\textwidth}@{}}
  \rule{\linewidth}{0.25pt}& \rule{\linewidth}{0.25pt}\\
  Ort, Datum & Unterschrift
  \end{tabular}
\end{center}

\subsection*{Belehrung}
Wer vorsätzlich gegen eine die Täuschung über Prüfungsleistungen betreffende Regelung einer Hochschulprüfungsordnung verstößt, handelt ordnungswidrig.
Die Ordnungswidrigkeit kann mit einer Geldbuße von bis zu \SI[round-mode=places, round-precision=2]{50000}{€} geahndet werden. 
Zuständige Verwaltungsbehörde für die Verfolgung und Ahndung von Ordnungswidrigkeiten ist der Kanzler/die Kanzlerin der Technischen Universität Dortmund. 
Im Falle eines mehrfachen oder sonstigen schwerwiegenden Täuschungsversuches kann der Prüfling zudem exmatrikuliert werden \mbox{(\S\,63 Abs. 5 Hochschulgesetz --HG--).}

Die Abgabe einer falschen Versicherung an Eides statt wird mit Freiheitsstrafe bis zu 3 Jahren oder mit Geldstrafe bestraft.

Die Technische Universität Dortmund wird ggf.\ elektronische Vergleichswerkzeuge (wie z.\,B.\ die Software \enquote{turnitin}) zur Überprüfung von Ordnungswidrigkeiten in Prüfungsverfahren nutzen. \\[\baselineskip]

\noindent Die oben stehende Belehrung habe ich zur Kenntnis genommen.\\[1cm]
\begin{center}
\begin{tabular}{@{}p{0.4\textwidth}@{\hspace{0.15\textwidth}}p{0.4\textwidth}@{}}
\rule{\linewidth}{0.25pt}& \rule{\linewidth}{0.25pt}\\
Ort, Datum & Unterschrift
\end{tabular}
\end{center}
